% Início do abstract
\begin{center}
    \textbf{ABSTRACT}
    \vspace{60pt}
\end{center}

Conceptual modelling is defined as a representation of an aspect of the reality, in which the relevant features for a context are evidenced to support some activity, to the detriment of the inherent complexity in the real object. The act of modeling requires that the modeler, in an ad hoc way, highlights the features of an object that are essentials in a certain context, however, not always these characteristics are evidenced with easiness. This master thesis proposes a process that intends to guide the modeler through phases in which he/she will be able to understand the activity that gives rise to a certain concept when faced with a specific object. As this process is based on the theories of the Linguistics, Philosophy and Psychology, it intends to help the modeler to discover the best possible concept to represent the knowledge that he/she wants to express. The process was evaluated through a case study whose models created using the process were compared with models that were created without using it. Our results suggest that the use of the process assists modelers to discover: (1) the knowledge concerning an object, (2) which characteristics are important of this object that allow categorizing it, (3) how they interact and give rise to the concept and, finally, its representation.

\vspace{20pt}

\textbf{Keywords:} Conceptual Modeling, Knowledge Representation, Conceptualization.
% Fim do abstract