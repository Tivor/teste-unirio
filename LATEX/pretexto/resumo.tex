% Início do resumo
Oliveira, Andre Gustavo Ferreira. \textbf{Um Processo de Modelagem Conceitual de Dados com Base em Teorias da Filosofia, Linguística e Psicologia.} UNIRIO, 2015. 94 páginas. Dissertação de Mestrado. Departamento de Informática Aplicada, UNIRIO.
\vspace{60pt}
\begin{center}
    \textbf{RESUMO}
    \vspace{60pt}
\end{center}

Modelagem Conceitual é definida como uma representação de um aspecto da realidade, onde as características relevantes para um contexto são evidenciadas para dar suporte a alguma atividade, em detrimento da complexidade inerente ao objeto real. O ato de modelar exige que o modelador, de maneira \textit{ad-hoc}, destaque as características de um objeto que são essenciais em um determinado contexto, porém essas características nem sempre se evidenciam com facilidade. Este trabalho propõe um processo que pretende guiar o modelador através de fases nas quais ele será capaz de entender a atividade que dá origem a um determinado conceito ao se deparar com um dado objeto. Por estar baseado nas teorias da Linguística, Filosofia e Psicologia, este processo pretende ajudá-lo a descobrir o melhor conceito possível para representar o conhecimento que o modelador quer expressar. O processo foi avaliado através de um estudo de caso, onde foram comparados modelos resultantes da sua utilização com modelos sem sua utilização. Nossos resultados sugerem que a utilização do processo auxilia modeladores a descobrirem: (1) como possuem o conhecimento acerca de um objeto, (2) quais características são importantes deste objeto que permitem a sua categorização e (3) como elas se relacionam, para dar origem ao conceito e, enfim, sua representação,Sistemas de Informação, Banco de Dados.

\vspace{20pt}

\textbf{Palavras-chave:} Modelagem Conceitual, Representação do Conhecimento, Conceitualização.
% Fim do resumo