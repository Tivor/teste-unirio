Este trabalho teve como foco a qualidade semântica e a qualidade pragmática de um modelo conceitual. Foi defendido que, para um modelo alcançar esses objetivos, o correto entendimento acerca das características que definem cada conceito, assim como suas relações, precisam ser entendidas e pensadas de maneira adequada. Para isso, é preciso reconhecer o objeto da reflexão, entender os processos que permitem apreender o objeto, contextualizar esse objeto, apontar quais as características deste objeto são relevantes neste contexto, buscar suas relações com outros objetos deste contexto, para então definir o símbolo que o representará.

Como apoio teórico, esse trabalho propõe um processo baseado nas teorias da filosofia, da psicologia e na linguística. A interseção destas três áreaspermitiu a reflexão na busca pelo processo de racionalização dos conceitos.

Outros trabalhos realizados na área da Ciência da Informação,atestam a significância e relevância das áreas utilizadas, seja individualmente ou em conjunto. Abaixo destacamos alguns trabalhos relevantes para nosso estudo e suas principais distinções da nossa abordagem.

Maria Luiza de Almeida Campos, (2001)apresenta uma metodologia de criação de hipertextos, onde os conceitos envolvidos na determinação de relacionamentos entre links devem ser considerados. Seu método está dividido em três níveis de entendimento e compreende sete requisitos básicos, onde questões como domínio do conhecimento, leitor final, identificação de conceitos e relacionamentos entre os nós, estabelecimento do veículo de comunicação são discutidas. O Processo Objeto-Representação embora esteja dividido em três fases, concentra-se na identificação das características que definem um determinado conceito no âmago do modelador e busca guiá-lo para que a representação desse conceito conserve todas as características e relacionamentos observados.

Em Bernardo Pereira Nunes et al. (2009), é apresentado uma técnica para classificação hierárquica automática de dados semiestruturados. Baseado na teoria de protótipo o trabalho buscou organizar os frames, através de seus atributos definidos como principais, em uma hierarquia. No Processo Objeto-Representação buscamos entender porque determinados atributos são tidos como principais em um dado contexto.

Rafael dos Santos Nonato; Gercina Ângela Borém de Oliveira Lima, (2012), buscou os aspectos importantes no tratamento da informação e da determinação de links em hipertextos. Em seu trabalho utilizando o protótipo MHTX, buscou em uma estrutura de navegação dividida em facetas encontrar novos relacionamentos para os links utilizando glossários e dicionários gerais. O Processo Objeto-Representação, neste aspecto busca os relacionamentos que um objeto possa ter com outros no próprio modelador e sua experiência com o contexto analisado, na busca da representação que permita o entendimento do conceito pelo maior numero de pessoas possíveis.

Naziozenio Antonio Lacerda, (2012)em seu trabalho mostrou-se interessado em entender qual seria a concepção dos educadores sobre tecnologia digital, buscando fundamentos que pudessem trazer novas perspectivas ao processo. No entanto, manteve-se focado na descrição de um único conceito. O Processo Objeto-Representação busca uma orientação que possa servir de base para auxiliar a descrição de contextos diversos.

Embora cada uma das áreas tenhacontribuições significativas, os trabalhos vistos não apresentaram um processo que abarcasse as três grandes áreas. Os resultados das avaliações indicam que a abordagem através do processo pode orientar os modeladores na descoberta do conceito mais adequado para expressar um objeto inserido em um contexto, originando um modelo conceitual semântico e pragmático.

6.1. Principais Contribuições

A principal contribuição deste trabalho é a apresentação de um processo para orientar a descoberta de conceitos em modelo conceitual. Esse processo, com aporte nas teorias da Filosofia, Psicologia e Linguística é estruturado em três fases, onde o resultado das reflexões de cada faseé o insumo para a fase seguinte.

Entendendo que a modelagem conceitual é uma área importante para a correta apreensão de conhecimento acerca de um domínio, o seu entendimento e sua representação fidedigna. Consideramos como boas práticas que poderiam auxiliar a utilização do Processo Objeto-Representação, em um primeiro momento a correta limitação do contexto acerca do qual se pretende emitir enunciados. Essa limitação, embora seja apresentada na Fase 02 seguindo o fluxo racional, trará maiores benefícios para os modeladores se forem observadas desde o primeiro momento, pois assim poderá definir quais conjuntos de palavras fazem mais sentido dentro do contexto em questão.

Afim de facilitar o processo de modelagem devemos, em um segundo momento, definir as maneiras através das quais interagimos com os objetos, isto é, quais são as formas de conhecer o mundo. Esse processo pode ser percebido ao nos depararmos com um objeto e buscarmos em nossa experiência objetos que possuam características semelhantes, objetos com quais interagimos ou utilizamos da mesma maneira, ou ainda objetos que apresentem forma final ou útil próxima. Esse entendimento nos levará naturalmente a enunciar cada uma das características que pudermos perceber no objeto, fornecendo uma lista que nos permitirá analisar quais destas características são necessárias para esse objeto seja relevante no contexto definido.

O processo de definição das características necessárias, envolve perceber como essas características se relacionam umas com as outras e, posteriormente, como o objeto como um todo se relaciona com outros objetos presentes no mesmo espaço contextual definido. Essa percepção permitirá encontrar a representação que, dentro do contexto, permitirá a transmissão mais adequada do conceito.

Essas recomendações geraispodem ser melhor visualizadas abaixo:

Delimitar o contexto no qual os objetos estão inseridos e através do qual serão representados.
Observar de maneira ampla os objetos que compõe o contexto, sem se preocupar com as representações que possam expressar os objetos apreendidos.
Distinguir de maneira clara quais sensações são despertadas pelo objeto, isto é, que sentidos são responsáveis pela apreensão do mesmo, ou se o objeto é um objeto inteligível.
Entender as maneiras pelas quais podemos conhecer esses objetos, descrevendo em enunciados curtos essas maneiras.
Organizar essas as características enunciadas de maneira a evidenciar suas relações umas com as outras e dos objetos entre si.
A utilização do processo permite:

A identificação de conceitos que de outra maneira poderiam ser identificados tardiamente.
A escolha adequada do símbolo para representar o conceito, trazendo clareza, consistência e codularidade para o modelo conceitual resultante.
Uma sistematização do processo de racionalização de um conceito, evidenciando o conhecimento acerca do objeto e sua aquisição.
A utilização do processo permite que linguagens distintas, utilizadas para a representação de contextos reais, sejam beneficiadas, poiso foco principal do processo está na escolha adequada dos conceitos e, o mais importante, em como saber que tal conceito é o mais indicado para evocar uma ideia. Tal propósito permite que sua utilização traga proveito para as linguagens de modelagens mais distintas, desde as mais simples como a Entidade-Relacionamento, até as mais expressivas.

6.2. Trabalhos Futuros

Para trabalhos futuros pretendemos analisar como as áreas da Antropologia, Inteligência Artificial e Neurociência se relacionam com as áreas exploradas neste trabalho, de acordo com o hexágono cognitivo apresentado em Gercina Ângela Borém Lima, (2003). Realizar mais experimentos utilizando um universo mais amplo, buscando modeladores com diferentes níveis de conhecimento, com o objetivo de ampliar os estudos sobre o processo e sua utilização.Será preciso analisar, sob o prisma das modelagens com aporte ontológicos,como a proposta porGiancarlo Guizzardi, (2005), se nosso processo é capaz de fornecer algum ganho aos modeladores e a esses processos. Pretendemos verificar se o Processo Objeto-Representação poderia apresentar ganho na modelagem realizada por grupos de modeladores, tanto de uma mesma área de conhecimento, como de áreas de conhecimentos distintas.

Aplicar o Processo Objeto-Representação em contexto diferente da modelagem conceitual, como o de buscas web por exemplo. Uma outra abordagem seria a criação ou adaptação de uma ferramenta que dê suporte as fases do processo.Outra possibilidade seria a utilização de mais abordagens teóricas acerca da aquisição,classificação, categorização e representação do conhecimento, afim de prover melhor aporte aos modeladores