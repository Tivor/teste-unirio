\chapter{Introdução}
\label{chap:introducao}

O homem desenvolve técnicas para representação de seu conhecimento, desde a antiguidade  \cite{nonato:2009.teoria}. A comunicação, pouco a pouco, foi se tornando possível com o advento da fala, fazendo da oralidade a principal via de compartilhamento do conhecimento. A escrita, ainda em seus primórdios, tentava trazer esse conhecimento para uma esfera atemporal e, posteriormente, interespacial \cite{silva:2011.fala}. 

Com expansão tecnológica, a partir do século XX, novas formas de representação do conhecimento surgiram, com objetivo de prover uma melhor maneira de capturar o mundo e seu conhecimento \cite{nonato:2009.teoria}. Diferentes autores, porém, ao longo dos séculos se ocuparam da tarefa se saber como se define o conhecimento, de como é possível ao homem conhecer e como o homem organiza esse conhecimento.

Com o surgimento da ciência cognitiva, na década de 60, pesquisas sobre cognição, natureza do conhecimento, funcionamento do cérebro e inteligência artificial, por exemplo, conseguiram avanços consideráveis, mostrando-se um campo promissor \cite{lacerda:2012.linguagem}.

\section{\hspace*{3pt} Justificativa}
\label{sec:justificativa}

A representação de conceitos não está isenta da subjetividade de quem o faz, nem do contexto na qual está inserida. Símbolos, despertam em seus observadores conceitos que podem remeter a objetos distintos em contextos distintos \cite{nonato:2009.teoria}.

O ato de modelar é um trabalho que exige atenção e dedicação, para que os objetos percebidos sejam representados, despertem os mesmos  conceitos, o mais próximo, em seus observadores que aqueles que criaram a sua representação. 

Autores como \citeonline{vickery:1980.classificacao}, \citeonline{campos:2004.modelizacao} e \citeonline{alvarenga:2003.representacao} consideram as representações como essenciais as atividades humanas de registro e produção de conhecimento.

Um modelador precisa entender seu contexto, para ser capaz de criar uma representação que transmita sua exata percepção. \citeonline{castro:2010.abordagem} aponta que um dos grandes problemas na modelagem conceitual é a dificuldade que o modelador possui de compreender os conceitos e que sua experiência e o conhecimento que ele acumulou, podem ser fatores preponderantes no processo.

As áreas da Linguística, Psicologia e Filosofia e suas teorias fortalecem o caráter multidisciplinar da área de Sistemas de Informação, além de oferecer um aporte teórico para a apreensão e entendimento dos objetos e conceitos que eles despertam \cite{lacerda:2012.linguagem}. 

Desta forma está pesquisa busca contribuir para o entendimento da possibilidade do conhecimento e sua representação, o que justifica a necessidade de encontrar um processo que ajude a construção de modelos cuja representação possua qualidade semântica e pragmática, através da abordagem mista vinda de diferentes áreas do saber e de como lidam com a questão acerca do conhecimento.

\section{Objetivo Geral}
\label{sec:objetivoGeral}

Analisar o processo de formação de conceitos e sua categorização, fundamentando em teorias da ciência cognitiva. Buscando compreender o processo desde a percepção do objeto até sua representação, e encontrar uma maneira de fornecer apoio para sua utilização no processo de descoberta de conceitos para modelagem conceitual.

\section{Objetivos Específicos}
\label{sec:objetivoEspecifico}

\begin{enumerate}
\item Partindo da percepção, identificar as características essenciais de um conceito e como identificamos essas características e a maneira como elas se relacionam.
\item Classificar e Categorizar os conceitos, de acordo com sua relação contextual, permitindo que a cada nova agregação  características ou conceitos, novas relações sejam determinadas permitindo a descoberta de novos conceitos.
\item Explicitar a formação dos conceitos em função de seus relacionamentos com outros objetos do contexto.
\item Propor um processo que possa guiar a descoberta de características e conceitos inerentes a um contexto.
\end{enumerate}

\section{Metodologia}
\label{sec:metodologia}

Este trabalho utiliza um método de natureza aplicada, com objetivo descritivo, através de uma abordagem qualitativa. Para desenvolver a pesquisa, adotaremos como modalidade o estudo de caso.

\section{Estruturação}
\label{sec:estruturacao}

Este trabalho está dividido seis capítulos. No capítulo 01 - Introdução, encontra-se a apresentação do tema da dissertação, a metodologia e a estrutura da dissertação. O capítulo 02 - Ciências Cognitivas, apresenta e maneira breve os assuntos que tangem está dissertação e lhe servem de fundamentação teórica. Trata a questão da percepção, da categorização e de como a linguística lida com a formulação de conceitos. O capítulo 03 - Modelagem Conceitual, trata da questão da representação do conhecimento, em especial através da abordagem de Modelos Conceituais. Neste capítulo também é descrito um estudo preliminar a fim de entender como se dá o processo através do qual os modeladores entendem e representam um contexto. No capítulo 04 - POR: Processo Objeto-Representação, apresentamos nossa proposta e o fluxo resultante do processo. No capítulo 05 - Avaliação, descrevemos as avaliações do processo, e finalmente no capítulo 06 - Conclusões são apresentadas algumas observações sobre o trabalho, bem como sugestões para trabalhos futuros.