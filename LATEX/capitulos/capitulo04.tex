A literatura demonstraque a racionalização do conhecimento é uma área fértil para perspectivas e abordagens distintas. No entanto acreditamos que, no âmbito da elaboração de modelos conceituais, uma maneira de representar o conhecimento, e perceber os pontos de convergência entre essas teorias pode fornecer um ponto de apoio no processo de representação conceitual acerca de um determinado domínio.

Neste propósito, defendemos que é preciso um processo que seja capaz de guiar os modeladores através dos pontos de convergência de maior destaque. Evidenciar a discussão inerente ao processo de construção de modelos conceituais, passando de maneira ativa por cada uma das etapas identificadas e discutidas nas abordagens que servem de fundamentação para este trabalho.

Com base nas teorias apresentadas neste trabalho, elaboramos um fluxo que visa auxiliar a elaboração de modelos conceituais. O processo proposto contempla fases que possuem como propósito guiar o modelador na busca acerca do seus conhecimentos sobre o contexto e como sucedeu sua aquisição. Tal busca, intenciona refletir na sua representação mais expressiva e completa.

Este processo, embora esteja dividido em três fases (Percepção, Racionalização e Representação), apresenta sua discussão centrada na fase 2, a racionalização, pois entendemos que esta fase contém o ponto crítico do processo de raciocínio e representação do conhecimento. Em cada fase, identificamos uma ou mais etapas que irão compor o fluxo, que serão apresentadas nas seções seguintes.

A figura 9 representa essas etapas resumidamente.


Fases do Processo Objeto-Representação



4.1. Fase 01: Percepção

A fase 01 foi definida como a Percepção, cuja função é fornececer os objetos que serão racionalizados.

Objeto é tudo aquilo que nos circunda ou é designado pelo homem; são as ``coisas" do mundo. Nesta dissertação, o termo "objeto" é empregado de maneira ampla e pode representar tanto objetos concretos (ex.: um animal ou um veículo), quanto objetos abstratos (ex.: um departamento ou um dragão)(Dahlberg, Ingetraut, 1978; Machado, Felipe Nery Rodrigues; Abreu, Maurício Pereira de, 2009).

Podemos, ainda, categorizá-los em objetos sensíveis, isto é, aqueles objetos que são evidenciados para nós através de um, ou mais, sentidos e objetos inteligíveis, cuja sensação é fruto de um processo mental, como por exemplo a junção de ideias de mulher e peixe, para formar uma sereia ou conceitos que não existem de maneira física, como sentimentos.

Ao categorizar os objetos em sensíveis e inteligíveis, apesar de reconhecer toda a batalha filosófica travada ao longo dos séculos entre as correntes denominadas Empirismo, Racionalismo e Criticismo, não estamos assumindo uma postura Empirista, mas queremos apenas evidenciar os objetos abstratos, como frutos de um processo racional e reflexivo, a despeito de quaisquer discussões filosóficas subjacentes.

A base da percepção humana são os sentidos, sendo estes estimulados de maneira contínua por um fluxo de acontecimentos. Disto resulta uma excitação neural chamada de sensação(Alexandre, Dulclerci Sternadt; Tavares, João Manuel R. S., 2007).

Sensação, segundo o dicionário onlinePriberam, (2015), é "a impressão produzida pelos objetos exteriores num órgão dos sentidos, transmitida ao cérebro pelos nervos, onde se converte em ideia, julgamento ou percepção".

Na introdução da Crítica da Razão Pura,Kant, Imanuel, (1983) afirma que:

"Não há dúvida de que todo o nosso conhecimento começa com a experiência; do contrário, por meio do que a faculdade de conhecimento deveria ser despertada para o exercício senão através de objetos que toquem nossos sentidos e em parte produzem por si próprios representações, em parte põem em movimento a atividade do nosso entendimento para compará-las, conectá-las ou separá-las e, desse modo, assimilar a matéria bruta das impressões sensíveis a um conhecimento dos objetos que se chama experiência? […]"(Kant, Imanuel, 1983).
Estas impressões, no entanto, são percepcionadas no espaço e no tempo, uma vez que formas puras (vazias) fazem parte das estruturas cognitivas inatas do sujeito. Elas são a condição indispensável para que possamos ter acesso ao conhecimento, isto é, a sensibilidade expressa-se em duas formas: espaço e tempo, os quais, nas palavras de Kant, Imanuel, (1983), são definidos, respectivamente, como:

"O espaço não é um conceito empírico abstraído de experiências externas. Pois a representação de espaço já tem que estar subjacente para certas sensações se referirem a algo fora de mim […] O espaço é uma representação a priori necessária que subjaz a todas as intuições externas. […] O espaço não é um conceito discursivo ou, como se diz, um conceito universal de relações das coisas em geral, mas sim uma intuição pura. […] O espaço é representado como uma magnitude infinita dada. […] A representação originária do espaço é, portanto, intuição a priori e não conceito" (Kant, Imanuel, 1983)
"O tempo não é um conceito empírico abstraído de qualquer experiência. […] O tempo é uma representação necessária subjacente a todas intuições. […] Sobre essa necessidade a priori também se funda a possibilidade de princípios apodíticos das relações do tempo, ou de axiomas do tempo em geral. […] O tempo não é um conceito discursivo ou, como se diz, um conceito universal, mas uma forma pura da intuição sensível. […] A infinitude do tempo nada mais significa que toda magnitude determinada do tempo só é possível mediante limitações de um tempo uno subjacente" (Kant, Imanuel, 1983)
O espaço e o tempo não são conceitos, visto que não são elaborados pelo sujeito tendo como ponto de partida suas experiências; eles simplesmente existem no sujeito cognoscente, que ao conhecer os objetos do mundo, o fazem de um modo dimensionado e associado à ideia de movimento, de mudança, de evolução, ou mesmo em estágios ou locais diferentes. Por isso as noções de espaço e de tempo são condições necessárias para o conhecimento dos objetos do mundo.

É através de sua percepção que um indivíduo organiza e interpreta suas impressões sensoriais para atribuir significado ao seu meio. Do ponto de vista cognitivo, a percepção envolve processos mentais que interpretam os dados oriundos dos sentidos e os associam aos conceitos.

4.2. Fase 02: Racionalização

A racionalização permite-nos dar significados aos estímulos que recebemos, transformando os dados oriundos das percepções em ideias ou conceitos.

Esta fase compartilha a visão atomística, que defende que a compreensão de um objeto implica o reconhecimento das partes para ter o entendimento do todo. Embora a visão holística - aquela que defende que a interpretação de um objeto se dá pelo todo - seja reconhecida, o todo quando inserido em um contexto, pode apresentar características que não sejam relevantes para a conceitualização do objeto.

Os experimentos propostos porRosch, Eleanor, (1999), para encontrar o nível médio de abstração: atributos comuns, movimentos motores, forma objetiva e semelhança da forma, ou as forma de conhecer o mundo, quando dentro do contexto, permitem que identifiquemos quais características são importantes para definição do objeto.

O contexto delimita, no universo do nosso discurso, quais as características dos objetos que percebemos fazem sentidos, e precisam ser compartilhadas para que nossas representações tenham sentidopara nossos interlocutores. 

A categorização reflete a nossa capacidade para agrupar entidades únicas em conjuntos, usando como regras de agrupamento as características que compartilham entre si.

4.2.1. Formas de Conhecer o Mundo

Atributos comuns são as características que dois ou mais objetos de uma mesma categoria compartilham. Ex.: ter bicos, ter penas, por ovos.

Movimentos motores representam a maneira como interagimos com determinados objetos, os movimentos musculares associados à utilização do objeto. Ex.: o ato de sentar em uma cadeira ou de cortar com um serrote.

Formas Objetivas e Similares dizem respeito à forma como o objeto se apresenta, isto é, a forma que ele possui. Ex.: cadeira de jantar, cadeira de praia, poltrona.

Nossa experiência associa características novas àquelas percebidas previamente, que sejam semelhantes. Refletir acerca da maneira como as características nos chegam nos ajudam a entender como estamos categorizando um determinado grupo de características.

4.2.2. Contexto

É o contexto que define o universo do discurso, influencia os conceitos que serão utilizados para expressarem um grupo de objetos ou relações. Segundo Olson, Hope A., (2012), a relevância do contexto para a nossa capacidade de interpretar o que percebemos não é uma fantasia filosófica acadêmica esotérica; É o que fazemos todos os dias. Ex.: Podemos falar de cinema, enquanto empresas de exibição de filmes de cinema; ou de cinema do ponto de vista da indústria que produz imagens com impressão de movimento, contendo narrativa.

Podemos afirmar então, que o ato de categorizar passa a ser encarado como um processo interacional, construído de maneira discursiva e dependente de um contexto(Carvalho, Maria de Lourdes Guimarães de; Souza, Mariléia de, 2013).

Vickery, Brian Campbell, (1980), citado por Silva, Alessandra Rodrigues da; Lima, Gercina Ângela Borém, (2011), reforçando essa ideia diz:

"A aquisição do conhecimento é um processo ativo. É uma interação concreta entre o organismo humano e seu ambiente, no curso do qual o ambiente é física e objetivamente mudado, e o organismo é também mudado, mas mental e subjetivamente. Estudando o desenvolvimento das categorias conceituais, não são apenas as atividades mentais como ‘a distinção’ que devem ser levadas em consideração. A atividade total, mental e física é envolvida (Vickery, Brian Campbell, 1980).
Medrado, Betânia Passos, (2008), afirma que o contexto traz além dos aspectos linguísticos, elementos corporais, gestuais, identidades institucionais e papéis sociais, ou seja, elementos socioculturais, produzindo uma relação dinâmica entre linguagem, cognição e interação(Carvalho, Maria de Lourdes Guimarães de; Souza, Mariléia de, 2013).

4.2.3. Categorização

Uma das formas que a categorização pode se dar, segundoRichardson, Roberto Jarry; Peres, José Augusto, (1985), é como o resultado da classificação progressiva dos elementos. A maior parte deste processo de categorização ocorre automática e inconscientemente, este processo só se torna perceptível a nós quando ocorrem casos dúbios. Oferecemos uma maneira de pensar sobre os conceitos e construir gradualmente o sistema de categorias que irão comportar os conceitos, conscientemente.

O significado linguístico está estreitamente relacionado com os processos de categorização, ``os conceitos, os significados não são, pois, rótulos das coisas nem objetos mentais aprioristicamente dados, mas categorias e, como tal, criações da cognição humana que servem para dar sentido ao mundo”(da Silva, Augusto Soares, 2006).

Categorizar é agrupar entidades (objetos, ideias, ações, etc) por semelhança(Lima, Gercina Ângela Borém, 2007). Este é um processo mental habitual ao homem, pois vivemos automaticamente classificando ideias e coisas a fim de compreender e conhecer a realidade(Piedade, Maria Antonieta Requião, 1983).

A Figura 10 desdobra a fase de Categorização/Conceitualização.


Fases do Processo Objeto-Representação (Estendido)
4.2.3.1. Características

Dahlberg, Ingetraut, (1978) define característica como uma unidade de conhecimento. Cada conceito é então entendido como um conjunto de características necessárias para a sua definição, e cada característica pode ser definida como um enunciado verdadeiro acerca do objeto(Dahlberg, Ingetraut, 1978).

Enunciado é qualquer frase ou oração, que exprima um pensamento de sentido completo, isto é, aquele pensamento que admite apenas um dos valores verdadeiro (V) ou falso (F). Mas para nossos propósitos, consideraremos apenas as frases declarativas cujo valor de verdade possa ser asseguradamente o verdadeiro (V).

Pelo fato de os enunciados sobre um objeto serem tão vastos, quanto nossos conhecimentos permitam, podemos tomar como exemplo a tabela categorial de Aristóteles, a fim de permitir a identificação do maior número de características possíveis.

As categorias Aristotélicas, apresentadas na Tabela 1, foram escolhidas como base neste trabalho, primeiro por entender a importância de sua classificação categórica para as teorias subjacentes e segundo por seguir a proposta deDahlberg, Ingetraut, (1978).

Categorias de Aristóteles
Categoria	Descrição	Exemplo
Matéria	É o que existe em si mesmo, o próprio objeto e seu material de origem	de pedra, de madeira, de vidro, etc.
Qualidade	É a determinação da matéria da substância, atribuindo-lhe partes distintas de outras partes.	possuir determinada estrutura, determinada forma, ser redondo, denso, colorido etc.
Quantidade (Extensão)	É a determinação da natureza ou da forma da substância.	possuir comprimento, largura, peso etc.
Relação	É a referência que um objeto ou uma característica possui com uma outra.	ser o dobro, ser mais largo, ser causa de, ser condição de, etc.
Processo (atividade, ação)	É o exercício das faculdades ou de poder sobre o objeto, de modo a produzir um efeito em alguma outra coisa ou nele mesmo.	começar, continuar, terminar, realizar algo etc.
Modo de ser	É posição relativa que as partes de um conceito têm quanto às outras.	estar em pé, sentado, voando, etc.
Passividade (paixão)	É a recepção sofrida, por um conceito, de um efeito produzido por algum agente.	ser cortado, pressionado, etc.
Posse (hábito)	Consiste em roupas, ornamentos ou outras posses.	usa sapatos, está armado, etc.
Localização (lugar, espaço)	É posição em relação aos corpos que circundam uma substância, que mede e determina o seu lugar.	estar em Brasília, no Rio de Janeiro, etc.
Tempo	É posição em relação ao curso de eventos extrínsecos, e que mede a duração de uma substância.	em fevereiro de 1978, etc.
A emissão de enunciados utilizando a tabela das categorias objetiva clarificar características que de outra maneira poderiam passar desapercebidas e evidencia, com maior facilidade, características simples. Sua utilização, no entanto, não tem a intenção de ser um fator limitante ou definitivo para a enunciação das características.

As formas de conhecer, propostas emRosch, Eleanor, (1999), justificam a apreensão do conhecimento para cada um dos enunciados acerca das categorias, de maneira que é possível relacionar cada categoria a pelo menos uma forma de conhecer, conforme Tabela 2.

Modos de Percepção do mundo x Categorias Aristotélicas

Atributos Comuns	Movimentos Motores	Formas Similares
Qualidade	X	X	X
Quantidade	X	X	X
Relação	

X
Processo	
X	
Modo de Ser	
X	
Passividade	
X	
Posse	
X	X
Localização	X	X	X
Tempo	X	X	X
Os enunciados podem conter características essenciais ou acidentais. Dahlberg, Ingetraut, (1978) define as características essenciais, como aquelas que definem o próprio conceito, por isso necessárias, e acidentais aquelas cuja a remoção não afetariam a categorização do objeto, por isso contingente.

Neste sentido, o contexto tem um papel importante, delimitando quais enunciados são essenciais para um determinado conceito. Embora a importância das características acidentais não sejam ignoradas, assim comoMedrado, Betânia Passos, (2008), consideraremos que apenas os enunciados necessários serão listados, quando o objeto está inserido no contexto.

Para evidenciar esse ponto, consideremos um carro (o objeto) que necessita de reparos elétricos (o contexto). Para este contexto, é irrelevante saber o tipo de combustível que o carro utiliza ou quantos quilômetros este mesmo carro é capaz de rodar com um litro de gasolina.

As características podem ainda ser simples ou complexas. Características simples dizem respeito a um único atributo. Ex.: azul, quadrado. Enquanto as complexas, usualmente, apresentam duas propriedades, ou mais, com alguma relação entre si. Ex.: pintado com tinta azul, moldado na forma quadrada. Em ambos os casos uma relação de processo(Dahlberg, Ingetraut, 1978).

4.2.3.2. Relações

Campos, Maria Luiza de Almeida, (2004) ressalta a importância das relações entre os objetos em um dado contexto, alegando que estas relações formam as estruturas conceituais deste contexto e que as mesmas possuem natureza diversa. Seguindo a definição deDahlberg, Ingetraut, (1978), para definição de conceitos, temos também a possibilidade de verificar as relações existentes entre as características de um conceito. Com esse propósito, ela define:

"Devemos estabelecer, desde logo, distinção entre as relações formais e as relações materiais, sendo que as primeiras se baseiam na comparação das características, tornando-se particularmente importantes quando se trata da compatibilidade dos conceitos e dos respectivos sistemas. As segundas têm por base o conteúdo das mesmas características. (Em síntese, deve ficar claro que as características são também conceitos, mas apenas em relação aos conceitos de que se tornaram elementos é que assumem o papel de características de conceitos)" (Dahlberg, Ingetraut, 1978).
Machado, Felipe Nery Rodrigues; Abreu, Maurício Pereira de, (2009) definem relacionamento como o fato, o acontecimento que une dois, ou mais, objetos do mundo real. As características atribuídas a diferentes conceitos nos guiará a uma análise acerca das relações entre estes conceitos.

A tabela 3, apresentada emDahlberg, Ingetraut, (1978), mostra os tipos de relacionamentos lógicos, através dos quais podemos definir as relações entre as características comuns. E através dos quais é possível estabelecer comparações entre os conceitos de modo a organizá-los.

Relacionamentos Lógicos
Relacionamento	Exemplo	Descrição
Identidade	A (x,x,x) e B (x,x,x)	As características são as mesmas;
Implicação	A (x,x) e B (x,x,x)	O conceito A está contido no conceito B;
Interseção	A (x,x,o) e B (x,o,o)	Os dois conceitos coincidem algum elemento;
Disjunção	A (x,x,x) e B (o,o,o)	Os conceitos se excluem mutuamente. Nenhuma característica em comum;
Negação	A (x,x,o) e B (o,x,o)	
O conceito A inclui uma característica cuja negação se encontra em B;

As relações lógicas desempenham um papel importante à medida que auxiliam a estabelecer comparações entre os conceitos, de modo a organizá-los nos seguintes relacionamentos semânticos: Relação de Hierarquia, Relação Partitiva, Relação de Oposição e Relação Funcional.

Relação Hierárquica (implicação) objetos que possuem características idênticas, porém um permite a ideia de um conceito mais geral que outro, então entre eles se estabelece a relação hierárquica ou relação de gênero e espécie. Pode-se então falar de conceitos mais amplos ou mais restritos. Pode-se também falar de conceito superior e inferior. O conceito superior é o mais genérico e o inferior é o mais específico(Dahlberg, Ingetraut, 1978).
Ex.: Mamífero → Cão → Pastor-Alemão

Relações hierárquicas, podem ser estabelecidas entre conceitos específicos do mesmo gênero, e recebem o nome de relações coordenadas, ou relações horizontais.
Ex.: Cão → Cão de Pequeno Porte → Pinscher, Chihuahua, Basset
.....Cão → Cão de Grande Porte → Fila, Labrador, Dálmata

Relação Partitiva existe entre dois ou mais conceitos, sendo um deles constituído por um outro. A observação de como um objeto se constitui, isto é, quais são suas partes (Campos, Maria Luiza de Almeida, 2001).
Ex.: árvore→ raízes, tronco, galhos, folhas, flores, frutos.

Relação de Oposição (negação) pode ser das seguintes espécies:
Contradição. Ex.: numérico - não numérico, presente - ausente
Contrariedade. Ex.: branco - preto

Relação Funcional (intersecção) estas relações aparecem quase exclusivamente na dependência do conceito de processo, ou seja, quando do conceito de processo deriva uma função a ele inerente. Ex.: Pintura (tem como consequência a existência de) quadros (que, por sua vez, supõe um) pintor (assim como de) críticos de arte (ou mesmo de) compradores de quadros, etc.

Será fácil verificar que as relações hierárquicas e as relações partitivas se aplicam principalmente a conceitos que expressam objetos. As relações de oposição se aplicam principalmente a conceitos que expressam características. Já as relações funcionais se aplicam sobretudo a conceitos que expressam processos(Dahlberg, Ingetraut, 1978).

4.2.3.3. Conceito

Antes mesmo da nossa capacidade da fala e da linguagem, já possuíamos a capacidade de atribuir significado às coisas. O conceito, desta forma, antecede a representação(Edelman, Gerald M, 1995) apud(Lacerda, Naziozenio Antonio, 2012). REVER

A formação do conceito dar-se-á em dois níveis, no nível individual onde as características são analisadas para que se defina os tipo de relações que mantêm entre si, e no nível contextual, onde um conceito é comparado com outro, então partindo de suas relações, sua correta intensão ou extensão seja definida, evidenciando o conceito mais adequado para representar.

Tomemos como exemplo um carro. Possivelmente a palavra escolhida para representar o conceito, por si só trouxe uma significância relevante. Porém listar as características deste único carro poderia ser uma tarefa extenuante e improfícua. Extenuante pois o número de enunciados que poderia proferir são inúmeros, e improfícuo porque sem determinar o contexto, esses enunciados podem não ter relevância. Desta forma, definiremos o contexto como sendo o de uma locadora de veículos.

Dentro deste contexto, considerando a atividade circunstancial de locar um carro, nosso escopo de características fica mais acessível. Onde podemos enunciar: seu modelo, cor, tipo de câmbio, quantidade de passageiros, tipo de combustível, tamanho da mala, quilometragem por litro, valor, se é aluguel por quilômetro rodado ou por dia, valor do seguro, etc. Apenas para citar algumas características.

A comparação do conceito carro com outros conceitos, talvez demonstre que na verdade seja melhor representar um veículo, ou ainda que esse conceito se relacionará com o funcionário e com o locatário, e que o funcionário talvez seja alguém mais específico.

4.3. Fase 03: Representação

A última etapa, a representação é a forma como escolhemos exteriorizar o conceito racionalizado. Usualmente essa forma de expressão é a linguagem e a escrita, mas não se limita apenas aelas.

A representação do conceito em uma linguagem que seja capaz de transmitir toda a sua carga de conhecimento é importante para que haja a correta representação do contexto e todo o conhecimento associado a ele.

A forma simbólica escolhida não faz parte das análises deste trabalho, mas entendemos que as fases anteriores, a fase 2 em especial, por levar ao conceito mais representativo para o contexto analisado, permitirá que linguagens com boas regras de formalização, que permitam a representação das unidades e suas relações possam ser utilizadas.