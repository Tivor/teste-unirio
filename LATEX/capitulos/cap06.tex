\externaldocument{capitulo06}

\chapter{\hspace*{3pt} Conclusão}
\label{chap:conclusao}

Este trabalho teve como foco a qualidade semântica e a qualidade pragmática de um modelo conceitual. Onde foi defendido que para um modelo alcançar esses objetivos, o correto entendimento acerca das características que definem o conceito, assim como suas relações precisam ser entendidas e pensadas de maneira adequada. Para isso, é preciso reconhecer o objeto da reflexão, entender os processos que permitem apreender o objeto, contextualizar esse objeto, apontar quais as características deste objeto são relevantes neste contexto, buscar suas relações com outros objetos deste contexto, para então definir o símbolo que o representará.

Como apoio teórico, esse trabalho propõe um processo baseado nas teorias da filosofia, da psicologia e na linguística. A interseção destas três áreas, permitiram a reflexão na busca pelo processo de racionalização dos conceitos.

Embora cada uma das áreas tenham contribuições significativas, tanto em abordagens individuais, quanto em abordagens conjuntas, os trabalhos vistos não apresentaram um processo que abarcasse as três grande áreas. Os resultados das avaliações indicam que a abordagem através do processo pode orientar os modeladores na descoberta do conceito mais adequado para expressar um objeto inserido em um contexto, originando um modelo conceitual semântico e pragmático.

\section{\hspace*{3pt} Principais Contribuições}
\label{sec:contribuicoes}

A principal contribuição deste trabalho é a apresentação de um processo para orientar a descoberta de conceitos em modelo conceitual. Esse processo, com aporte nas teorias da Filosofia, Psicologia e Linguística é estruturado em três fases, onde o resultado das reflexões cada fase, é o insumo para a fase seguinte. A utilização do processo permite:

\begin{itemize}
\item A identificação de conceitos que de outra maneira poderia ser identificado tardiamente.%, evitando retrabalho.
\item A escolha adequada do símbolo para representar o conceito, trazendo clareza, Consistência e Modularidade para o modelo conceitual resultante.
%\item A possibilidade de modelos conceituais mais próximos à realidade representada.
\item Uma sistematização do processo de racionalização de um conceito, evidenciando o conhecimento acerca do objeto e sua aquisição.
\end{itemize}

A utilização do processo permite que linguagens distintas utilizadas para a representação de contextos reais, sejam beneficiadas pois seu principal foco está escolha adequada dos conceitos e o mais importante, como saber que tal conceito é o mais indicado para evocar uma ideia. Tal propósito permite que sua utilização traga proveito para as linguagens de modelagens mais distintas, desde as mais simples como a Entidade-Relacionamento, até as mais expressivas. 

\section{\hspace*{3pt} Trabalhos Futuros}
\label{sec:trabalhosFuturos}

Nosso primeiro passo é a realização de mais experimentos, como os aqui relatados, utilizando um universo mais amplo, buscando modeladores com diferentes níveis de conhecimento, com o objetivo de ampliar os estudos sobre o processo e sua utilização. Uma outra abordagem seria a criação ou adaptação de uma ferramenta que dê suporte as fases do processo. Outra possibilidade seria a utilização de mais abordagens teóricas acerca da aquisição, classificação, categorização e representação do conhecimento, afim de prover melhor aporte aos modeladores.